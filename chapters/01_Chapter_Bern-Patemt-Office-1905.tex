%%
% Chapter_Bern-Patemt-Office-1905.tex
%
% @author: Nikola Stankovic
%%

\section{Bern Patent Office, 1905}

\emph{Hermann Einstein}, Albert Einstein's father, wrote to \emph{Professor Wilhelm Osterwald} to please him to write a letter to Albert with a few words of encouragement, so that be might recover the joy of Einstein in living and working. Einstein felt profoundly unhappy with his lack of position in that time and his idea that he has gone off the tracks with his career. No answer from Professor Osterwald was ever received.

In the year 1905, Einstein wrote a series of papers that changed our view of the universe forever.

Einstein married a fellow student, \emph{Mileva}, and worked in the patent office. He spent his time often in pub visits and long walks. Einstein's final university grades were unusually low. Teachers were irritated by his lack of obedience. Everyone in authority seemed to enjoy putting Einstein down.

Einstein and his wife had given away their first child, a daughter born before they were married. He couldn't even afford the money for part-time help to let his wife go back to her studies.

Even the hours he had to keep at the patent office worked against him. By the time he got off for the day, the one science library in Bern was usually closed. During the few free moments, he scribbled on sheets he kept in one drawer of his desk - which he jokingly called his "department of theoretical physics".

The first articles he wrote weren't especially impressive. He was always aiming for grand linkages

Einstein wrote his theory of relativity in just five or six weeks filling thirty-some pages. He sent his articles to \emph{Annalen der Physik} to be published. A few weeks later he realized that he had left something out, so he delivered a supplement.